
%% This work is distributed under the LaTeX Project Public License (LPPL)
%% ( http://www.latex-project.org/ ) version 1.3, and may be freely used,
%% distributed and modified. A copy of the LPPL, version 1.3, is included
%% in the base LaTeX documentation of all distributions of LaTeX released
%% 2003/12/01 or later.
%% Retain all contribution notices and credits.




\documentclass[times]{elsarticle}
%\IEEEoverridecommandlockouts



\usepackage{bsymb}
%\usepackage[top=2.5cm, bottom=2.0cm, left=2.5cm, right=2.5cm]{geometry}

\usepackage{cite}
\usepackage{amsmath,amssymb,amsfonts}
\usepackage{algorithmic}
\usepackage{graphicx}
\usepackage{subfig}

\usepackage{textcomp}
\def\BibTeX{{\rm B\kern-.05em{\sc i\kern-.025em b}\kern-.08em
    T\kern-.1667em\lower.7ex\hbox{E}\kern-.125emX}}

\usepackage{mathtools}


\usepackage[utf8]{inputenc}
\usepackage[T1]{fontenc}
\usepackage{microtype}

\usepackage{color}
\definecolor{gray}{rgb}{0.4,0.4,0.4}
\definecolor{darkblue}{rgb}{0.0,0.0,0.6}
\definecolor{cyan}{rgb}{0.0,0.6,0.6}
\definecolor{keycolor}{rgb}{0,0,0.8}     % Keywords, blue
\definecolor{labelcolor}{rgb}{0,0.4,0.8} % Labels, cyan
\definecolor{codecolor}{rgb}{0,0,0}      % Formulas, black
\definecolor{inhcolor}{rgb}{0.6,0.2,0}   % Inherited formulas, maroon
\definecolor{cmtcolor}{rgb}{0,0.4,0}     % Comments, green

\usepackage{bsymb}

\usepackage{listings}

\usepackage{color}
\definecolor{gray}{rgb}{0.4,0.4,0.4}
\definecolor{darkblue}{rgb}{0.0,0.0,1.0}
\definecolor{cyan}{rgb}{0.0,0.6,0.6}


\lstdefinelanguage{MPS}
{
  morestring=[b]",
  morestring=[s]
 % {>}{<}
  ,
  morecomment=[s]{/*}{*/},
  stringstyle=\color{black},
  identifierstyle=\color{black},
  keywordstyle=\color{darkblue},
  morekeywords={domain,model,parent,concepts,concept,is,variable,
  individuals,deduced,attribute,domain,dom,relations,relation,attributes,
  enumerated,elements,
  range,functional,total,maplets,custom,data,sets,set,values,value,type,lexical,form,
  predicates,p1,p2,not}% list your attributes here
  ,
  otherkeywords = {=,&,(,),\{,\},>,<,",:,?},
}

\lstset{language=MPS}




\usepackage{lineno,hyperref}
\modulolinenumbers[5]

\journal{Arxiv}


\usepackage{breqn}
\usepackage{longtable}



\usepackage[linewidth=1pt]{mdframed}
\usepackage{multicol}

\usepackage{flushend}




\begin{document}

\begin{frontmatter}

\title{Towards Certified Ethical Artificial Intelligences
}
%\tnotetext[mytitlenote]{French National Research Agency (ANR)\n Natural Sciences and Engineering Research Council of Canada  (NSERC)}
%\subtitle{Do you have a subtitle?\\ If so, write it here}

%\titlerunning{Short form of title}        % if too long for running head


% author names and affiliations
% use a multiple column layout for up to three different
% affiliations
\author{Steve Tueno}
\address{Université de Sherbrooke, Québec, Canada}
\ead{steve.jeffrey.tueno.fotso@usherbrooke.ca}






%\maketitle

\begin{abstract}
This paper is about a proposal to allow the definition of ethical artificial intelligences for which ethics can be certified.
The proposal is based on definition of ethics ontologies, each ontology aiming at formalizing some ethical principles of interest. 
During the learning phases of machine learning algorithms, the  ethics ontologies will be exploited  in order to reason on learning data and build ethically acceptable machine learning models.
For instance, an ontology may describe that learning can only be achieved if the learning data is almost equitably balanced between men and women.
Thus, a learning algorithm may link possible values of a coefficient to be determined to either the percentage of data for which the sex field is set to 'F' or to the ones for which the field is set to 'M'.
Machine learning models generated following the proposal / algorithms / tools will be certified ethically correct, which will have an impact on their adoption and usage.

\end{abstract}

\begin{keyword}
Artificial Intelligence \sep Machine Learning \sep Ethical Correctness \sep  Ontologies
\end{keyword}

\end{frontmatter}

%\linenumbers



\section*{Introduction}



\section{Background}






\section*{References}
% trigger a \newpage just before the given reference
% number - used to balance the columns on the last page
% adjust value as needed - may need to be readjusted if
% the document is modified later
%\IEEEtriggeratref{8}
% The "triggered" command can be changed if desired:
%\IEEEtriggercmd{\enlargethispage{-5in}}

% references section

% can use a bibliography generated by BibTeX as a .bbl file
% BibTeX documentation can be easily obtained at:
% http://mirror.ctan.org/biblio/bibtex/contrib/doc/
% The IEEEtran BibTeX style support page is at:
% http://www.michaelshell.org/tex/ieeetran/bibtex/
%\bibliographystyle{IEEEtran}
% argument is your BibTeX string definitions and bibliography database(s)

%% `Elsevier LaTeX' style
\bibliographystyle{elsarticle-num}
\bibliography{references}

\end{document}
% end of file template.tex


